\documentclass{article}
\usepackage{amsmath}
\usepackage{geometry}
\geometry{margin=1in}

\title{SIMM Sensitivities Ageing}
\author{ISDA SIMM Implementation}
\date{}

\begin{document}
\maketitle

\section*{Purpose}
This note documents the ageing methodology implemented in
\texttt{src/sensivities\_ageing.py}. The goal is to take a spot (0D) CRIF file
and roll sensitivities forward to each SIMM tenor bucket in
\texttt{simm\_tenor\_list}.

\section*{Tenor to Day Conversion}
Each SIMM tenor is converted to a day count using:
\[
2w = 14,\quad m = \frac{365}{12},\quad y = 365.
\]
This aligns with the time scaling convention already used in the curvature
scaling helper.

\section*{Ageing Rule}
For a sensitivity with original tenor $T_{\text{orig}}$ (in days) and an ageing
horizon $T_{\text{age}}$ (also in days), we compute:
\[
T_{\text{rem}} = T_{\text{orig}} - T_{\text{age}}.
\]
If $T_{\text{rem}} \le 0$, the sensitivity is considered matured and is removed.
Otherwise, the remaining tenor is bucketed to the nearest lower (or equal)
SIMM tenor bucket:
\[
T_{\text{bucket}} = \max\{t \in \text{SIMM buckets} : t \le T_{\text{rem}}\}.
\]
This ensures we do not extend the maturity when rolling down sensitivities.

\section*{Handling of Different Risk Types}
The ageing algorithm applies to any CRIF row that uses \texttt{Label1} for
tenor classification (e.g., IR, Credit, Equity, Commodity, and FX where
relevant). Rows with non-tenor \texttt{Label1} values are left unchanged.

\section*{Outputs}
The function returns a dictionary of dataframes:
\begin{itemize}
  \item \textbf{0D}: the original spot CRIF.
  \item \textbf{Each SIMM tenor}: the CRIF with rolled \texttt{Label1} buckets and
        matured sensitivities removed.
\end{itemize}

\end{document}
